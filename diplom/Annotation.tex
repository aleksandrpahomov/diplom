% Шапка документа - не менять
\documentclass[12pt, a4paper]{extarticle}
\usepackage{amsfonts}
%\usepackage[T2A]{fontenc}
\usepackage[cp1251]{inputenc}
\usepackage{mathtext}  
\usepackage{amsmath, amsfonts, amssymb}
\usepackage[russian]{babel}
\usepackage[body={17.5cm, 23.5cm},left=2cm, top=2cm]{geometry}
\sloppy
\binoppenalty=10000
\relpenalty=10000


% Определения некоторых макросов - не менять

% Макросы для ученых степеней
\newcommand{\kfmn}{{\mdseries канд.~физ.-мат.~наук}}
\newcommand{\ken}{{\mdseries канд.~эконом.~наук}}
\newcommand{\kpn}{{\mdseries канд.~пед.~наук}}
\newcommand{\khn}{{\mdseries канд.~хим.~наук}}
\newcommand{\dpn}{{\mdseries доктор~пед.~наук}}
\newcommand{\dfmn}{{\mdseries доктор~физ.-мат.~наук}}
%ученые звания
\newcommand{\doc}{{\mdseries доцент}}
\newcommand{\prof}{{\mdseries профессор}}

%Заголовок аннотации
\newcommand{\atitle}[1]{\begin{center}{\Large #1\par}\end{center}}
%автор
\newcommand{\auth}[2]{\noindent{\bf #1}, #2 курс\addcontentsline{toc}{subsection}{#1}\par}
%научный руководитель
\newcommand{\swise}[1]{\noindent Научный руководитель: {\bfseries #1\par}}
\newcommand{\coswise}[1]{\noindent Соруководитель:  #1\par}


% Текств
\begin{document}

% Название работы, тема курсовой, диплома.
\atitle{Анализ поведения нелинейной динамической системы в одном критическом случае}

% Фамилия Имя Отчество автора и курс, на котором выполнена работа.
\auth{Бушуева Татьяна Эдисоновна}{4}

% ученая степень, ученое звание Фамилия и инициалы научного руководителя
% Ученые степени (расшифровки есть выше):
% \kfmn, \ken, \kpn, \khn, \dpn, \dfmn
% если у научного руководителя иная степень, то оформить ее словами - новый макрос заводить не нужно
% Ученые звания (расшифровки есть выше):
% \doc, \prof
% Если у научного руководителя отсутствует ученая степень и/или звание, то не пишется ничего
\swise{\dfmn, \prof~Кубышкин Е.П.}

\medskip

В $\mathbb{R}^6$ рассматривается нелинейное дифференциальное уравнение
\begin{equation}\label{formula1}
\dot {x} = A(\varepsilon) x + F(x),
\end{equation}
$A(\varepsilon)$ - матрица, $F(x)$ - вектор-функция $(F(0) = 0, \|F(x)\| = O(\|x\|))$ гладко зависят от своих аргументов при $0 \leqslant \varepsilon \leqslant \varepsilon_0,
\| x\| \leqslant x_0.$
Предполагается, что собственные значения матрицы $A(\varepsilon)$ имеют вид
\begin{equation*}
\lambda_j(\varepsilon) = \tau_j(\varepsilon) \pm i\omega_j(\varepsilon) \;\; (\tau_j(0) = 0, \omega_j(0) = \omega_j > 0, \;  i = \sqrt{-1}, \; j = 1,2,3),
\end{equation*}

а между величинами $\omega_j$ выполняется резонансное соотношение
\begin{equation*}
\omega_1 = 2\omega_2 - \omega_3 .
\end{equation*}

Изучается поведение решений уравнения в фиксированной окрестности нулевого состояния равновесия в зависимости от параметров уравнения.

Выполнена нормализация системы уравнений  $(\ref{formula1})$. В результате имеем систему уравнений
\begin{align}\label{formula2}
&\dot{z}_j = (\lambda_j(\varepsilon) + a_{j1}(\varepsilon)|z_1|^2 + a_{j2}(\varepsilon)|z_2|^2 + a_{j3}(\varepsilon)|z_3|^2)z_j + \\ \notag
& + A_j(\varepsilon)z_1^{k_{j1}}z_2^{k_{j2}}z_3^{k_{j3}} + O(|z|^4) \\
& ( j = 1,2,3 ; \; A_1(\varepsilon)z_2^2\overline{z}_3, A_2(\varepsilon)z_1\overline{z}_2z_3, \;\; A_3(\varepsilon)\overline{z}_1z^2_2). \notag
\end{align}

Здесь $z_j = x_j + iy_j$ - комплекснозначные переменные $ (j = 1,2,3)$, $|z|^2 = |z_1|^2 +|z_2|^2 +|z_3|^2$, $a_{jk}(\varepsilon) = b_{jk}(\varepsilon) + i c_{jk}(\varepsilon)$ гладкие функции $0 \leq \varepsilon \leq \varepsilon_0.$  

В $(\ref{formula2})$ выполнена замена переменных $z_j = \rho_j e^{i\omega_j\tau_j} \; (j = 1,2,3)$, введены "медленные"\ переменные $\rho_j \;\; (j = 1,2,3), \;\; \Theta = -\omega_1\tau_1 + 2\omega_2\tau_2 - \omega_3\tau_3 $ и  "быстрые"\ переменные  $\tau_1, \tau_2 $. 
В системе "медленных"\ переменных выполнены нормирующие замены $\rho_j\rightarrow \varepsilon^{1/2}\rho_j$ и время $t \rightarrow \varepsilon^{-1} t$. 

Относительно коэффициентов "главной части"\ уравнения $(\ref{formula2})$ в "медленных"\ переменных  сделано предположение о выполнении условий теоремы А.М. Молчанова$[1]$ (условие диссипативности).

Показано, что в уравнениях "медленных"\ переменных возможно существование различных автоколебательных решений, в том числе и хаотических. Для хаотических решений вычислены показатели Ляпунова и ляпуновская размерность.

\medskip

{\large \noindent Список литературы}

1. Молчанов А.М. Устойчивость в случае нейтральности линейного приближения. ДАН СССР, 1961, т.141, № 1, с. 27.



\end{document}
